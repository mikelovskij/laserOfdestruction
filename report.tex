\documentclass[a4paper,11pt]{article}

\usepackage[utf8x]{inputenc}
\usepackage[english]{babel}
\usepackage{amssymb}
\usepackage{amsmath}
\usepackage[dvips]{graphicx}
%\usepackage{floatflt}
%\usepackage{psfrag}
\usepackage{subfigure}
\usepackage{booktabs}
\usepackage{ctable}
\usepackage{siunitx}
\usepackage[a4paper,centering,margin=1.3in,text={7in,10in},bindingoffset=0mm]{geometry}
\usepackage{amsmath}
\usepackage{textcomp}
\usepackage[version=3]{mhchem}
\usepackage{multicol}
\usepackage{hyperref}
\usepackage{epstopdf}

\usepackage[usenames,dvipsnames]{pstricks}
\usepackage{epsfig}
\usepackage{pst-grad} %
\title{Oxygen Photoacoustic effect using an external cavity diode laser}
\author{Michele Valentini, Zeno Tornasi, Henry Widodo}



\usepackage{pst-plot} % For axes
\makeindex  
\begin{document}
\maketitle
\newpage

\begin{abstract}

In this report we will write about the experiment done during the first and second weeks of August in order to measure one "line"` in the visible part of the absorption spectrum (ATMOSPHERIC BAND/B-BAND?) of the Oxygen molecule using the photoacoustic effect. In order to excite the oxygen molecular orbitals we used a diode laser tuned with an external cavity. INSERT SMALL RESUME OF THE REPORT? FOR EXAMPLE (IN CHAPTER 1 WE WILL TALK ABOUT BLABLA, IN CHAMPTER 2 ABOUT BLIBLI ETC?)
\end{abstract}
\newpage
\section{Experimental apparatus}
The experimental apparatus is composed by:
\subsection{The laser source}
We used an external cavity laser device, formed by the following elements:
\begin{itemize}
\item a single mode multi-quantum well AlGaInP laser diode\footnote{Hitachi HL6738MG: \url{http://pdf.datasheetcatalog.com/datasheets/50/502031_DS.pdf}}.
 The lasing wavelength can be tuned from about 680 nm to 695 nm by adjusting the driving current and the diode temperature.
\item a temperature controller case\footnote{Thorlabs TCLDM9: \url{http://www.thorlabs.de/Thorcat/1900/TCLDM9-Manual.pdf}} to set the temperature of the diode. 
\item an external cavity, i.e. a setup that feeds back the laser diode with the first diffraction order of a 1800 grooves/mm grating. The external cavity allowed us to better select a given lasing mode, thus getting a smaller emission linewidth. The grating was put on a piezoelectric mechanical actuator, which permitted nm-order adjustments of its position. Since a grating diffracts different frequencies at different angles, moving the grating we could control the frequency fed back to the laser and thus enhanced. This is how we got a fine tuning of the frequency, and how we were able to make the HOWMANYGHz scan to see the absorption line. 
%Since the linewidth of the oxygen resonance peak is of the order ofmagnitude 5 GHz and
\end{itemize} 
\subsection{The acoustic chamber} 
 The gas to analyze, pure O$_2$ at atmospheric pressure, was contained in a chamber 17 cm long. Four microphones were put halfway in the chamber, one for each side of it. A strain gauge vacuometer\footnote{Edwards ASG-1000-NW16:\vspace{-10pt} \begin{flushright} \url{http://www.ultimatevacuum.dk/D35725880\%20ASG\%20user\%20manual.pdf}\end{flushright}} was in the chamber as well.
The laser light enters the cavity through transparent windows. Being absorbed, it heats up the oxygen thus generating pressure waves. By chopping the laser light at a frequency which is resonant for the chamber, we enhance those waves and make them become a sound detectable by the microphones.
\subsection{section2}
\end{document}




