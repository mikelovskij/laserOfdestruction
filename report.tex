\documentclass[a4paper,11pt]{article}

\usepackage[utf8x]{inputenc}
\usepackage[english]{babel}
\usepackage{amssymb}
\usepackage{amsmath}
\usepackage[dvips]{graphicx}
%\usepackage{floatflt}
%\usepackage{psfrag}
\usepackage{subfigure}
\usepackage{booktabs}
\usepackage{ctable}
\usepackage{siunitx}
\usepackage[a4paper,centering,margin=1.3in,text={7in,10in},bindingoffset=0mm]{geometry}
\usepackage{amsmath}
\usepackage{textcomp}
\usepackage[version=3]{mhchem}
\usepackage{multicol}
\usepackage{hyperref}
\usepackage{epstopdf}

\usepackage[usenames,dvipsnames]{pstricks}
\usepackage{epsfig}
\usepackage{pst-grad} %
\title{Oxygen Photoacoustic effect using an external cavity diode laser}
\author{Michele Valentini, Zeno Tornasi, Henry Widodo}



\usepackage{pst-plot} % For axes
\makeindex  
\begin{document}
\maketitle
\newpage

\begin{abstract}

In this report we will write about the experiment done during the first and second weeks of August in order to measure one "line"` in the visible part of the absorption spectrum (ATMOSPHERIC BAND/B-BAND?) of the Oxigen molecule using the photoacoustic effect. In order to excite the oxygen molecular orbitals we used a diode laser tuned with an external cavity. INSERT SMALL RESUME OF THE REPORT? FOR EXPAMPLE (IN CHAPTER 1 WE WILL TALK ABOUT BLABLA, IN CHAMPTER 2 ABOUT BLIBLI ETC?)
\end{abstract}
\newpage
\section{Experimental apparatus}
The experimental apparatus is composed by two main parts:
\paragraph{The external cavity laser}
In order to find one absorption peak of the $O_2$ molecule, we used a tunable laser. The laser is composed(O SI USA SOLO PER GLI SPARTITI?) by a single mode multi-quantum well AlGaInP laser (HL6738) diode which is inserted in a temperature controlled mount (TCLDM9).Controllong the temperature and the current of the laser diode allows to tune the laser emission wavelenght about from  680 nm to 695 nm. In order to get a smaller linewidth and since we need to be capable of fine tuning the output vawelenght we put in front of the laser a diffraction grating with 1800 grooves per millimeter. This grating reflects the first diffraction order of a selected wavelenght back inside the laser and "'forces"` it to emit in the mode which contains that wavelenght. 
%Since the linewidth of the oxygen resonance peak is of the order ofmagnitude 5 GHz and 
\paragraph{The acoustic cavity/chamber } 
 (?which is the technical term?) in which the laser is sent.



something
\subsection{section2}
\end{document}




