\documentclass[a4paper,11pt]{article}

\usepackage[utf8x]{inputenc}
\usepackage[english]{babel}
\usepackage{amssymb}
\usepackage{amsmath}
\usepackage[dvips]{graphicx}
%\usepackage{floatflt}
%\usepackage{psfrag}
\usepackage{subfigure}
\usepackage{booktabs}
\usepackage{ctable}
\usepackage{siunitx}
\usepackage[a4paper,centering,margin=1.3in,text={7in,10in},bindingoffset=0mm]{geometry}
\usepackage{amsmath}
\usepackage{textcomp}
\usepackage[version=3]{mhchem}
\usepackage{multicol}
\usepackage{hyperref}
\usepackage{epstopdf}

\usepackage[usenames,dvipsnames]{pstricks}
\usepackage{epsfig}
\usepackage{pst-grad} %
\title{Oxygen Photoacoustic effect using an external cavity diode laser}
\author{Michele Valentini, Zeno Tornasi, Henry Widodo}



\usepackage{pst-plot} % For axes
\makeindex  
\begin{document}
\maketitle
\newpage

\begin{abstract}

In this report we will write about the experiment done during the first and second weeks of August in order to measure one "line"` in the visible part of the absorption spectrum (ATMOSPHERIC BAND/B-BAND?) of the Oxygen molecule using the photoacoustic effect. In order to excite the oxygen molecular orbitals we used a diode laser tuned with an external cavity. INSERT SMALL RESUME OF THE REPORT? FOR EXAMPLE (IN CHAPTER 1 WE WILL TALK ABOUT BLABLA, IN CHAMPTER 2 ABOUT BLIBLI ETC?)
\end{abstract}
\newpage
\section{Experimental apparatus}
The setup we used features the typical photoacoustic experiment characteristics. There was a source of light, a laser in our case, impinging on the gas into a cavity. A mechanical chopper provided a modulation of the light in order to match a proper frequency of the cavity. The acoustic signal, detected by microphones, was filtered by a lock-in amplifier referenced by the chopping frequency. Several optical elements, such as mirrors, lenses and beam-splitters were used to control the light. Standard laboratory instruments, such as generators, waveform generators, oscilloscopes were used as well. We'll now describe in more details the main elements of the apparatus.
\subsection{The laser source}
We used an external cavity laser device, formed by the following elements:
\begin{itemize}
\item a single mode multi-quantum well AlGaInP laser diode\footnote{Hitachi HL6738MG: \url{http://pdf.datasheetcatalog.com/datasheets/50/502031_DS.pdf}}.
 The lasing wavelength can be tuned from about 680 nm to 695 nm by adjusting the driving current and the diode temperature.
\item a temperature controller case\footnote{Thorlabs TCLDM9: \url{http://www.thorlabs.de/Thorcat/1900/TCLDM9-Manual.pdf}} to set the temperature of the diode. 
\item an external cavity, i.e. a setup that feeds back the laser diode with the first diffraction order of a 1800 grooves/mm grating. The external cavity allowed us to better select a given lasing mode, thus getting a smaller emission linewidth. The grating was put on a piezoelectric mechanical actuator, which permitted nm-order adjustments of its position. Since a grating diffracts different frequencies at different angles, moving the grating we could control the frequency fed back to the laser and thus enhanced. This is how we got a fine tuning of the frequency, and how we were able to make the HOWMANYGHz scan to see the absorption line. 
\end{itemize} 
\subsection{The acoustic chamber} 
 The gas to analyze, pure O$_2$ at atmospheric pressure, was contained in a brass chamber, featuring :
\begin{itemize}
\item an internal cavity, about \mbox{13 cm} long, where the gas actually resonates. Other two smaller cavities are present before and after the main cavity. Since we couldn't open the brass chamber, we had no way to accurately measure the dimensions and the position of the main cavity with respect to the other two ones. It should be noticed that our chamber has been recycled from another experiment and was not explicitly thought for the usage we do of it. However, there were two marks on the outside of the chamber, that were supposed to indicate where the main cavity started and ended.
\item four microphones put about halfway in the chamber, one for each side of it. Due to the fact that the chamber couldn't be opened, we don't know whether the microphones were actually halfway. According to the marks, they were not. In fact, they were 5 mm away from the supposed middle point.
\item an active strain gauge vacuometer\footnote{Edwards ASG-1000-NW16:\vspace{-10pt}
\begin{flushright} \url{http://www.ultimatevacuum.dk/D35725880\%20ASG\%20user\%20manual.pdf}
\end{flushright}} measured the pressure in the chamber.
The laser light entered the cavity through two circular transparent windows.
\end{itemize}
\subsection{section2}
\end{document}

 The oxygen absorbs the light and heats up locally, giving rise to pressure waves. By chopping the laser at a chamber proper frequency we enhance resonant acoustic waves and make them become a sound strong enough to be detected by the microphones.


