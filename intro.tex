\chapter*{Introduction}
\addcontentsline{toc}{chapter}{Introduction}

This report presents our final experience for the course of Experimental Physics (advanced), held by prof. M. Scotoni, University of Trento, during academic year 2012/13.

\bigskip
The task was to observe at least one line in the the visible part of the absorption spectrum of \ce{O2} at ambient conditions, using the photoacoustic effect. As a starting point, we have been given a photoacoustic chamber and a red laser diode to excite the oxygen with, together with a case and a controller to adjust the diode temperature and driving current.

\bigskip
In \cref{chapter1} the apparatus and the measurement instruments are globally described, while in \cref{chapter2} we expose some solutions to optimize the apparatus for the measurement we needed to do, as well as some issues we encountered in setting up the most delicate parts.

\medskip
In \cref{chapter3} the collected measurements are presented and discussed. \cref{chamber} deals with the search of the chamber resonance, \cref{oxygen} with the search of the oxygen peaks. \cref{shapeaks} is about the actual observation of the peaks and the study of their shape. \cref{freedrift} contains a small analysis we did on the behaviour of the laser diode. The results in there are important to understand where some effects we encountered across all the previous sections come from.

\medskip
\cref{ECDL,lokkin,etalon} are intended to provide a basic theoretical background about three key elements of our apparatus: the ECDL, the lockin and the etalon.